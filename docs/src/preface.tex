\chapter*{Introduction}
\addcontentsline{toc}{chapter}{Introduction}
In this project we study a Multi-agent pickup \& delivery (MAPD) problem 
where multiple moving agents are placed into a known environment where they are
given potentially infinite stream
of tasks -- deliveries. In order for an agent to execute a task, it first has to
reach a pick-up location and then carry a package to a delivery location in a
collision-free manner. The
MAPD problem is to assign tasks to agents so that the average time of all 
deliveries is minimal. It is a well studied problem which is known to be
NP-hard, so approximations and heuristics are developed to address variants of
the pick-up \& delivery problems. 
% We can distinguish MAPD solutions into two
% categories. One is centralized where one ``entity'' is assigning packages to
% all agents. The other one is decentralized, where every agent decides for 
% himself what he's going to do. 

\vspace{5mm}\noindent
We will study MAPD solutions on grids that will be
similiar to real world environments, for example grids with tight corridors just
like in warehouses with many shelves next to each other. Big part of these environments
are obstacles which have to be avoided by agents. These obstacles can act as
buildings if the environment is a city or pillars if the environment is a
warehouse.
In one step agents area ble to move
in four directions -- up, down, left or right.


\vspace{5mm}\noindent Every task is characterized by
its pickup location, delivery location and a release time. We will study an
online version of the MAPD problem, where package pick-up and delivery locations
are known once they are released and not before. This leads to a possibility for
agents to change their scheduled tasks but only when they are not already carrying
a package. We forbid dropping packages anywhere other than at the delivery location
since it is more applicable in real world scenarios. 
We can see direct analogies of these scenarios to real 
world problems. Examples include robots in automated warehouses, 
aircraft-towing vehicles, video-game characters or even taxis who pick up and
drop off passengers.\cite{PDP}\cite{Liu}

\section*{Technologies}
In software project I will focus on creating a simulation tool for this problem
using Python programming language, more specifically \textit{Pyglet} graphics
library. I will focus on Linux platform, but final program should work on any
operating system thanks to \textit{Pyglet}. Another library that I will use is
\textit{argparse}, which is a very famous Python library for parsing
command-line arguments. I will also use \textit{python3-tk} (tkinter) library
which allows to open windows of host operating system to open or save files.
This will be used for loading and saving environments. Also I will use python
virtual environment for better package organization and avoidance of conflicts
with packages in my main python directory.

\newpage
\section*{Graphical user interface and settings}
The simulation tool will contain a graphical interface which will be rendering 
agents as they perform their tasks. A goal will be to create a robust tool, making
it easily adjustable for similiar multi-agent pathfinding problems. User will be
able to create or edit environment by drawing using mouse or load existing
environment from a file. Environments will be created on rectangular grids.
I will try to implement features that will
make an analysis ``by eye'' easier -- for example, a user will be able to track an
agent, which will display currently planned path for a selected agent. Also,
information about state of the simulation (current average delivery time, 
number of occupied agents,
delivered packages,..) will be displayed in the control panel aswell.
Agents will be respresented as circles distinguished by colors.
Agents with assigned tasks will be marked by a small circle on top of their main
representational circle. 

\section*{Algorithms}
TODO.
\section*{Limitations}
TODO.
% In the second part of this project I will implement some existing solutions to 
% this problem and compare them against each other and against my own solution as well.
% I will compare these models on grids representing various real life instances. 


\vspace{5mm}\noindent In a potential following bachelor's thesis I would review
existing MAPD solutions and compare them to my own solution through experimental
evaluation in a greater detail.
